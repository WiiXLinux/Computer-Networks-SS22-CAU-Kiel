\documentclass[12pt, a4paper]{article}

\usepackage[english]{babel} 
\usepackage[T1]{fontenc}
\usepackage{amsfonts} 
\usepackage{setspace}
\usepackage{amsmath}
\usepackage{amssymb}
\usepackage{listings}
\usepackage{xcolor}

\newcommand*{\qed}{\null\nobreak\hfill\ensuremath{\square}}
\newcommand*{\puffer}{\text{ }\text{ }\text{ }\text{ }}

\pagestyle{plain}
\allowdisplaybreaks

\definecolor{codegreen}{rgb}{0,0.6,0}
\definecolor{codegray}{rgb}{0.5,0.5,0.5}
\definecolor{codepurple}{rgb}{0.58,0,0.82}
\definecolor{backcolour}{rgb}{0.95,0.95,0.95}

\lstdefinestyle{mystyle}{
    backgroundcolor=\color{backcolour},   
    commentstyle=\color{codegreen},
    keywordstyle=\color{magenta},
    numberstyle=\tiny\color{codegray},
    stringstyle=\color{codepurple},
    basicstyle=\ttfamily\footnotesize,
    breakatwhitespace=false,         
    breaklines=true,                 
    captionpos=b,                    
    keepspaces=true,                                   
    showspaces=false,                
    showstringspaces=false,
    showtabs=false,                  
    tabsize=2
}

\lstset{style=mystyle}


\title{Computer Networks - Assignment 7}
\author{Jannis Kühl, Henri Heyden\\ \small stu241399, stu240825}
\date{}


\begin{document}
\maketitle
\section*{Task 3}
\subsection*{a) and b)}
The full table should look like this:
\begin{lstlisting}
No. Time  Source          Destination         Protocol Info
1 0.0000 	0.0.0.0:68	 	  255.255.255.255:67 	DHCP 	DISCOVER
2 0.0003 	10.44.12.33:67	255.255.255.255:68 	DHCP 	OFFER	
3 0.0005 	10.44.0.10:67 	255.255.255.255:68 	DHCP 	OFFER	
4 0.0701 	0.0.0.0:68	 	  255.255.255.255:67	DHCP 	REQUEST	  
5 0.0702 	10.44.12.33:67 	255.255.255.255:68 	DHCP 	ACK		 
\end{lstlisting}
The two DHCP servers are 10.44.12.33 and 10.44.0.10.
\subsection*{c)}
The obvious reason is that the client doesn't have an address yet, so the server messages have to be broadcast. \\
It could also be useful, because there are two DHCP servers:\\
When the first server ACK's the request, the other server knows that this address is taken, so it can remove this address from their pool. \\
This of course only makes sense if the two DHCP servers share some addresses of their pool of possible addresses.

\end{document}