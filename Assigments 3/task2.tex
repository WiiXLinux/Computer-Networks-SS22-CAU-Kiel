\documentclass[12pt, a4paper]{article}

\usepackage[english]{babel} 
\usepackage[T1]{fontenc}
\usepackage{amsfonts} 
\usepackage{setspace}
\usepackage{amsmath}
\usepackage{amssymb}
\usepackage{pgf-umlsd}
\usepackage{titling}

\newcommand*{\qed}{\null\nobreak\hfill\ensuremath{\square}}
\newcommand*{\puffer}{\text{ }\text{ }\text{ }\text{ }}
\newcommand*{\lhop}{\mathrel{\overset{\makebox[0pt]{\mbox{\normalfont\tiny\sffamily l'hop.}}}{=}}}

\pagestyle{plain}
\allowdisplaybreaks
\setlength{\droptitle}{-10em}

\title{Computer Networks - Assignment 3}
\author{Jannis Kühl, Henri Heyden\\ \small stu241399, stu240825}
\date{}


\begin{document}

\vspace{-5cm}
\maketitle

\section*{Task 2}
\subsection*{a)}
The following graphic should illustrate the delays for one HTTP v1.0 request. \\
Here we will name the given delays with \(d_t := 3s\) being the servers output transmission delay and \(d_{RTT} := 2s\) being the RTT-delay. \\
As you can see I am not good with \LaTeX... \\ \\
\makebox[380pt]{
\begin{sequencediagram}
    \newinst{A}{Client}{}
    \newinst[2]{B}{Server}{}

    \mess[1]{A}{}{B}
    \node [anchor = east] (t0) at (mess from) {\(t = 0s\): Initiate TCP connect at p=80};
    \node [anchor = west] (t0) at (mess to) {Accept connection at port 80.};

    \mess[1]{B}{}{A}
    \node [anchor = west] (t0) at (mess from) {Notify client.};
    \node [anchor = east] (t0) at (mess to) {Get notified.};

    \mess[1]{A}{}{B}
    \node [anchor = east] (t0) at (mess from) {\(t = 2s = d_t\): Send request message.};
    \node [anchor = west] (t0) at (mess to) {Read request message.};

    \mess[3]{B}{200: OK + file}{A}
    \node [anchor = west] (t0) at (mess from) {Send response message and close TCP connection.};
    \node [anchor = east] (t0) at (mess to) {\(t = 7s = 2d_t + d_{RTT}\): File received};
\end{sequencediagram}
}
\subsection*{b)}
Since the process illustrated in \textbf{a)} will have to be done 100 times, since we're using HTTP v.1.0 and are restricted to only one active TCP connection per client, we can calculate the total time: \(T = 100 \cdot t_1 = 700s\). Since we only want the time between the first request being transmitted and the last file arriving at the client, we'll subtract \(d_{RTT}\) one time and get the asked time which is 698 seconds. \\
\end{document}