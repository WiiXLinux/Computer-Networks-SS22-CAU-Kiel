\documentclass[12pt, a4paper]{article}

\usepackage[english]{babel} 
\usepackage[T1]{fontenc}
\usepackage{amsfonts} 
\usepackage{setspace}
\usepackage{amsmath}
\usepackage{amssymb}

\newcommand*{\qed}{\null\nobreak\hfill\ensuremath{\square}}
\newcommand*{\puffer}{\text{ }\text{ }\text{ }\text{ }}
\newcommand*{\lhop}{\mathrel{\overset{\makebox[0pt]{\mbox{\normalfont\tiny\sffamily l'hop.}}}{=}}}

\pagestyle{plain}
\allowdisplaybreaks

\title{Computer Networks - Assignment 5}
\author{Jannis Kühl, Henri Heyden\\ \small stu241399, stu240825}
\date{}


\begin{document}
\maketitle
\section*{Task 2}
\subsection*{a)}
In UDP we would have to demultiplex manually, but because we know that we're analyzing HTTP requests here, there is some sort of transport-layer-demultiplexing, since http relies on TCP.\\
With that we know that the part needed to demultiplex incoming request is the 4-tuple of (source IP-Address, source port number, destination IP-Address, destination port number). \\
This information is stored in the IP part of the packet and the UDP part of the incoming packet. \\
Since this tuple is individual for every client, the transmission layer can distinguish between request even if destination port, destination address and source port are the same.\\
(Notice that if a client would make two requests at the same time these two requests would be handled individually, because the source port can't be the same.)
\subsection*{b)}
As already mentioned, with transport-layer-demultiplexing the server can distinguish between every request. \\
In this case as mentioned by the assignment, source port of the request coming from client A and the source port of the "second" request from client C are the same. But notice that the source IP are not the same, and since the transport-layer-demultiplexing works with the 4-tuple we mentioned in a), we can distinguish between both requests. \\
We can even distinguish between the two requests by client C since the source ports are different.
\end{document}